\section{Conclusion and Future Work}\label{sec:conc}
Real-world engineering design problems may often require computationally intensive simulation-based optimization~(SO). In some cases, SO techniques may allow the fidelity of output results to be specified, with lower-fidelity solutions consuming less of the allowed computational budget while providing a coarse(r) estimate of the performance.

This paper presented a multi-fidelity simulation-based optimization algorithm designed to solve bound-constrained, single-objective problems called \AlgName{}. It is an iterative, two-stage process, built on the commonly-used co-kriging method, which updates its model by employing a modified version of the optimal computing budget allocation algorithm which samples from a restricted neighbourhood. \hemant{The key differentiators of the proposed approach compared to the literature are that [...]} 

In order to demonstrate the effectiveness of the improved sampling technique of \AlgName{}, numerical experiments were conducted on two separate datasets, one multi-fidelity dataset from the literature, and one dataset made by applying a generic method used to model simulation errors to two standard test functions. The results of obtained by \AlgName{} were compared with those from a base-line co-kriging  \motos{} algorithms.  The experiments on the first dataset demonstrated the versatility of \AlgName{} on a variety of test functions, while those on the second dataset investigated the effects of problem size and error type on \AlgName{}.

The results of the numerical experiments confirmed that \AlgName{} produced solutions that were better than \motos{} and competitive with, or superior to, the base-line co-kriging algorithm across all instances for a limited computational budget. The main finding illustrated by the results is that by sampling the search space more selectively when updating its surrogate model, \AlgName{} is more efficient in its computing budget use, resulting in faster convergence rates --- especially as the size of the problem increases.

Further research in this direction could include investigating the use of different global search techniques to optimize the co-kriging model and applying it to a wider range of test functions and real-world problems, including problems with integer and categorical decision variables, and linear and non-linear constraints.