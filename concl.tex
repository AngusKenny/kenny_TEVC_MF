\section{Conclusion and Future Work}\label{sec:conc}
Real-world engineering design problems can be very complex, requiring computationally intensive simulation-based techniques to optimize (SO). One property of SO techiniques is that they often allow the fidelity of output results to be specified, with lower-fidelity solutions comsuming less of the allowed computational budget.

This paper presented a multi-fidelity simulation-based optimization algorithm designed to solve bound-constrained, single-objective problems called \AlgName{}. It is an iterative, two-stage process, built on the commonly-used co-kriging method, which updates its model by employing a modified version of the optimal computing budget allocation (OCBA) algorithm which samples the search space from a restricted neighbourhood.  

In order to demonstrate the effectiveness of the improved sampling technique of \AlgName{}, numerical tests were carried out that compared it to a base-line co-kriging implementation, which updates its model using random sampling. The experiments were conducted on two separate datasets, one multi-fidelity dataset from the literature, and one dataset made by applying a generic method used to model simulation errors to two standard test functions. The experiments on the first dataset demonstrated the versatility of \AlgName{} on a variety of test functions, while those on the second dataset investigated the effects of problem size and error type on \AlgName{}.

The results of the numerical experiments confirmed that \AlgName{} produced solutions that were competitive with, or superior to, the base-line co-kriging algorithm across all instances. The main finding illustrated by the results, is that by sampling the search space more selectively when updating the co-kriging model, \AlgName{} is able to use its computing budget more efficiently resulting in faster convergence rates, especially as the size of the problem increases.

Further research in this direction could include investigating the use of different global search techniques to optimize the co-kriging model and applying it to a wider range of test functions and real-world problems, including problems with integer and categorical decision variables, and linear and non-linear constraints.