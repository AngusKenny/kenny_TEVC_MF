\section{Conclusion and Future Work}\label{sec:conc}
Real-world engineering design problems often involve computationally intensive simulation-based analysis. In some cases, the fidelity of such simulations can be controlled i.e. a computationally less intensive and less accurate lower-fidelity analysis can be used instead of a computationally expensive and more accurate analysis.

In this paper a multi-fidelity simulation-based optimization algorithm is presented that can solve bound-constrained, single-objective problems called \AlgName{}. It is based on an iterative, two-stage process and relies on commonly-used co-kriging method coupled with a modified version of the optimal computing budget allocation algorithm that samples from a restricted neighbourhood. The key feature of the proposed approach is its means to enable a two-way information sharing. The information gained from the high-fidelity evaluations is used to determine the restricted neighbourhood from which the low-fidelity samples are selected. This allows for more efficient use of the computing budget as compared to contemporary approaches.

In order to demonstrate the effectiveness of the improved sampling technique of \AlgName{}, numerical experiments were conducted on two separate test suites, one multi-fidelity test suite from the literature, and one test suite created by applying a generic method used to model simulation errors of two standard test functions. The results  obtained by \AlgName{} were compared with those from a base-line co-kriging and \motos{} algorithms.  The experiments on the first suite demonstrated the versatility of \AlgName{} on a variety of test functions, while those on the second test suite demonstrated the effects of problem size and error type on the performance of \AlgName{}.

The results of the numerical experiments confirmed that \AlgName{} produced solutions that were better than \motos{} and competitive with, or superior to, the base-line co-kriging algorithm across all instances for a limited computational budget. By sampling the search space more selectively when updating its surrogate model, \AlgName{} was more efficient in its use of computing budget, resulting in faster convergence rates --- especially for higher-dimensional problems.

Further research in this direction could include investigating the use of different global search techniques to optimize the co-kriging model and applying it to a wider range of test functions and real-world problems, including problems with integer and categorical decision variables, and with linear and non-linear constraints. Some of these directions are currently pursued by the authors.