\section{Introduction}
% The very first letter is a 2 line initial drop letter followed
% by the rest of the first word in caps.
% 
% form to use if the first word consists of a single letter:
% \IEEEPARstart{A}{demo} file is ....
% 
% form to use if you need the single drop letter followed by
% normal text (unknown if ever used by the IEEE):
% \IEEEPARstart{A}{}demo file is ....
% 
% Some journals put the first two words in caps:
% \IEEEPARstart{T}{his demo} file is ....
% 
% Here we have the typical use of a "T" for an initial drop letter
% and "HIS" in caps to complete the first word.

% \hemant{IEEE requires use of US versions of the spelling ('-ize').}
\IEEEPARstart{M}{any} real-world engineering design problems require estimation of responses that are intractable for exact or analytical methods. In such cases, two alternatives are commonly resorted to: numerical simulations and physical testing of a prototype~\cite{forrester2008engineering}. When used in-loop during design optimization using iterative methods such as evolutionary algorithms~(EAs), both of these methods tend to be prohibitively slow; as well as potentially cost and resource intensive~\cite{jin2009systems}. Such optimization problems where each design evaluation incurs significant cost in any form are referred to as computationally expensive optimization problems.

Simulation-based optimization~(SO) refers to the methods that deal with optimization problems involving  numerical simulations\footnote{For brevity, the discussion is restricted to simulation-based design, but the same principles can be applied to other expensive optimization such as those involving physical prototyping.}. Typically, these methods make use of approximations, also referred to as metamodels/surrogates, to reduce the computational expense~\cite{amaran2016simulation}. The basic principle is to use historical information from previously evaluated designs to build a surrogate model of the response landscape. The predicted values from these surrogate models can then be utilized to determine the performance of new candidate solutions and only the competent ones can then be evaluated using the true~(time consuming) simulation. Through this informed pre-selection, the number of expensive evaluations can be significantly reduced during the process of optimization. 

Some numerical simulation processes allow control over the resolution, or fidelity of the analysis. For example, in finite element analysis~(FEA) or computational fluid dynamics~(CFD) simulations, the mesh size can be controlled to yield solutions with different fidelities~\cite{branke2016efficient,toal2015some}. A coarse mesh yields a low-fidelity~(LF) performance estimate that is relatively fast but less accurate, while a fine mesh yields a high-fidelity~(HF) estimate that is relatively more time consuming but more accurate. Multi-fidelity methods~(MF) are a special class of methods that attempt to efficiently combine the information of obtained from evaluation in various fidelities with an eventual aim to identify the optima of highest fidelity~(HF) response function. 

A number of different approaches MF approaches have been explored in the literature. Many of these approaches are based on the co-kriging technique described by Forrester et al.~\cite{forrester2007multi} which correlates the two sets of data, \ray{use HF and LF instead of exensive and cheap} and expensive, to produce a single prediction model. This technique is an extension of the autoregressive model first introduced by Kennedy and O'Hagan~\cite{kennedy2000predicting}. The two key aspects in which the different methods in this category deviate from each other, is in how they collect these data and also how they search using the model for potential candidate solutions. Laurenceau and Sagaut~\cite{laurenceau2008building} investigated a number of different sampling methods for use with kriging and co-kriging for airfoil design problems; Huang et al.~\cite{huang2013research} combined a genetic algorithm with co-kriging to optimize wing-body drag reduction in aerodynamic design; Perdikaris et al.~\cite{perdikaris2015multi} incorporated elements of statistical learning into a co-kriging framework to cross-correlate ensembles of multi-fidelity surrogate models; Yang et al.~\cite{yang2019physics} adopted a physics-informed approach, constructing models based on sparsely observed domain knowledge, representing unknowns as random variables or fields which are regressed using elements of co-kriging; and Giraldo et al.~\cite{giraldo2020cokriging} provided an extension to co-kriging for use when the secondary variable is functional, based on the work of Goulard and Voltz~\cite{goulard1993geostatistical}.

Among the approaches that do not use co-kriging models, some of the prominent ones include the following: Lv et al.~\cite{lv2021multi} employed a canonical correlation analysis-based model, in which the least squares method was used to determine optimal parameters; Ariyarit and Kanazaki~\cite{ariyarit2017multi} used a hybrid method which employed a kriging model to estimate local deviations and a radial basis function to approximate the global model in an airfoil design problem; Hebbal et al.~\cite{hebbal2021multi} and Cutajar et al.~\cite{cutajar2019deep} used machine learning techniques that treat the layers of a deep Gaussian process as different fidelity levels to capture the non-linear correlations between the fidelities; Xu et al.~\cite{xu2016mo2tos} used a two-stage process with the first relying on ordinal transformation to transform the original multi-dimensional design space into a one-dimensional ordinal space and then sampled from this ordinal space using a method based on the optimal computational budget allocation (OCBA) algorithm proposed by Chen and Lee~\cite{chen2011stochastic}; Branke et al.~\cite{branke2016efficient} and Lim et al.~\cite{lim2008evolutionary} both adopted evolutionary approaches to solving MF optimizations problems; Bryson and Rumpfkeil~\cite{bryson2018multifidelity} introduced a framework with a quasi-Newton method which can be used with many different surrogate models; and Ng and Willcox~\cite{ng2014multifidelity} proposed a number of approaches to deal with uncertainity in multi-fidelity optimization problems.

Many of these approaches sample the low and high-fidelity data \emph{a priori} \ray{please dont use the term data, you can say samples}, then build models based on these data and use some global search method to optimize them. To ensure these constructed models are properly representative, it is important to maintain a diversity of samples across the entire the design space; however, sampling without any prior knowledge can result in many computational resources being expended in areas which do not contain any promising candidates. Of those which do sample the data iteratively, information is typically only shared in one direction between the two datasets. Low-fidelity data is sampled randomly, or using some indepenent process, and used to inform where the high-fidelity data should be sampled from; but no information is then shared in the reverse direction, to inform the low-fidelity sampling in the next iteration. Again, this can result in computational resources being consumed unnecessarily in regions of the search space which are unproductive, especially in high-dimensional problems. 


To overcome these limitations and improve the performance for MFSO, this paper proposes an iterative two-stage, bound-constrained, single-objective multi-fidelity optimisation problems, referred to here as \AlgName{}. It uses previously obtained information about promising areas of the search space to define a restricted neighbourhood using a guided differential evolution (DE) process on a kriging model of the low-fidelity data. This neighbourhood is then sampled from and searched, using a method derived from OCBA, to determine a set of candidates to undergo low-fidelity simulation. The information from these simulations is used to update the low-fidelity model and also a co-kriging-based surrogate model of the high-fidelity data, which is searched globally using DE to find a suitable candidate for high-fidelity simulation. Finally, these high-fidelity data is used to update the surrogate model and also to help determine the restricted neighbourhood in the next iteration. By using the high-fidelity simulation information to inform and restrict the region of interest while searching the low-fidelity model, \AlgName{} allows two-way information sharing between the sets of data. The performance of the \AlgName{} model is compared against a baseline co-kriging-based MFSO algorithm on two separate datasets. The first is a common set of multi-fidelity test-functions from the literature, and the second is a set of multi-fidelity test functions that are generated from standard test functions using the methods described in the paper by Wang et al. In addition to this, some important properties of \AlgName{} are also investigated.

The remainder of this paper is organized as follows. Section~\ref{sec:back} provides the fundamentals and background of the proposed model, along with a description of the type of problems tackled and related work. The \AlgName{} algorithm is detailed in Section~\ref{sec:method}, describing all of its constituent parts and detailing some similarities and differences with a related technique from the literature. Experimental design and datasets are discussed in Section~\ref{sec:exp}, with Section~\ref{sec:results} giving the results of these experiments and a discussion of their implications. Finally, Section~\ref{sec:conc} provides the conclusion and outlines any future directions of research.

\ray{I suggest MFO instead of MF and MFSO}